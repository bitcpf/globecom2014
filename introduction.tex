\section{Introduction}
\label{sec:introduction}

% Background multiband 
% Channel utility
% Traditional hypothesis in previous works
% White space benefit in rural and challenge in populated area
% Issues
% Paper organization

Numerous cities are pursing city-wide WiFi deployment since the 
beginning of 21st century. Wireless Mesh Network is an ideal solution
for the coverage target. Both academies and industries are putting 
efforts on providing Internet connection for metropolitan areas~\cite{pahlavan2011principles}.
While a few mesh networks have been deployed in 
certain populated communities~\cite{CRSK06,google_imc08},
wireless mesh networks have largely been unsuccessful in achieving 
the scale of what was once anticipated~\cite{taps}. 
As a result, many network carriers opted to pay millions of dollars 
in penalties rather than facing the exponential-increasing deployment
 costs~\cite{cnet_aug07}.
Part of the reasons are the cost of building tons of access points 
and the failure of cooperation with existing wireless instrument.

%http://news.cnet.com/8301-10784_3-9768759-7.html
%http://arstechnica.com/gadgets/2008/05/philadelphias-municipal-wifi-network-to-go-dark/
Around the same time, the digital TV transition created more
spectrum for using with data networks~\cite{fccwhitespace}. These white 
space bands operate in available channels from 54-806 MHz, having
increased propagation characteristics as compared to 
WiFi~\cite{balanis2012antenna}. The large scale communication range 
help the network carriers reducing the number of access points for 
covering a certain area. Hence, the FCC has identified rural
areas as a key application for white space networks since the reduced
population from major metropolitan areas allows a greater service area
per backhual device without saturating wireless capacity. 
Naturally, the question arises for these rural communities as well as more dense 
urban settings: {\it how can the emerging white space bands improve 
large-scale mesh network deployments?} 
While much work has been done on deploying wireless networks
the differences in propagation and the coexisting activities in white space bands
have not been exploited simultaneously~\cite{si2010overview}.
Recognizing the coexisting signals in both WiFi channels and white space band channels 
is an prerequisite for applying wireless network in all the topics.
Previous work has investigate multi-channel, multi-radio wireless network in 
gateway placement, channel assignment, and routing problems. However, 
these works fail to involve white space band in their wireless network.
Moreover, the clean channel hypothesis held in these works fails to match the
in-field spectrum utility, which could slash the performance of mesh 
network in their models.

% Paper topic
In this paper, we apply in-field measurement spectrum utility across WiFi bands and white
space bands in multiband scenario to leverage the diversity in propagation 
and spectrum utility in the deployment of large-scale wireless mesh networks. 
To do so, we first form a metric quantify spectrum utility. 
Second, we proposed a measurement driven framework for finding the number of 
access points for multiple type of dense areas. Then, with in-field measured spectrum utility
data in Dallas-Fort Worth area we tell the activity level in these WiFi and white space bands. 
We apply our framework with the measurement results and shows the band selection 
variation in downtown area, neighborhood, university campus, urban area and rural area.
Finally, we analyze the white space band channels in band combination performance improve 
for a typical urban area.

% Paper contributions
The main contributions of our work are as follows:
\begin{itemize}
\item We design and perform in-field spectrum utility measurement in DFW metropolitan.
 Then we analyze the measurement results in multiple type dense areas.
\item We develop a measurement driven multi-band wireless mesh deployment framework 
to jointly leverage white space and WiFi bands for serving wireless mesh networks in an arbitrary area.  
\item We perform extensive analysis across diverse propagation, existing spectrum utility under capacity
 and coverage constraints with our measurements and framework, showing that with white space band, 
 the number of access points outperforms WiFi only deployment by up to 1650\% in sparse rural area.
\item Given similar channel resources, we additionally analyze the white space band performance in multiple
combination of channels. 
\end{itemize}


