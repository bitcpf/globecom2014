\section{Introduction}
\label{sec:introduction}

% Dployment constraints: QOS, capacity; coverage; resource constraint
% Beamforming
% Problem: beamforming in differet QoS environments
% Beamforming benefit in network deployment
% Issues
% Paper organization


% Background multiband, propagation diversity
The FCC has approved the use of broadband services in the white spaces of 
UHF TV bands, which were formerly exclusively licensed to television broadcasters.
These white space bands are now available for unlicensed public use, enabling the
deployment of wireless access networks across a broad range of scenarios from 
sparse rural areas (one of the key applications identified by the FCC) to dense urban 
areas~\cite{carlson}. The white space bands operate in available channels from 
54-806 MHz, having a far greater propagation range than WiFi bands for similar
transmission power~\cite{balanis2012antenna}. 

Specific to rural areas, the lack of user density and corresponding traffic
demand per unit area as compared to dense urban areas allows greater levels of
spatial aggregation to reduce the total number of required access points, lowering
network deployment costs. In densely populated urban areas, the greater concentration
of users and higher levels of traffic demand can be served by maximizing the spatial
reuse. While many works have worked to address multihop wireless network deployment
in terms of maximizing served user demand and/or minimizing network costs,
the unique deployment heterogeneous access points of white space bands and 
WiFi bands have either not been studied~\cite{si2010overview}. 
Specifically, previous work has investigated wireless network deployment in terms of 
multiradio network, power control, gateway placement, channel assignment, 
and routing~\cite{kodialam2005characterizing,he2008optimizing,marina2010topology}.
However, each of these works focus on the deployment in WiFi bands without
considering the white space bands. The white space band could extend the capacity 
degree and the coverage degree of an access point simultaneous. 

In WiFi and white space heterogeneous wireless network, the service area degree of 
an access point depends on the capacity of radios, the propagation range and the 
demands of the serving area. The scant frequencies of radios, the propagation distinctive
and the demands diversity of population distribution bring the variation of an access point
service area. These issues are substantial to designing an optimal network deployment 
and provide potential commercial wireless services to clients in any location.

Thus, the new opportunities created by white spaces motivate the following 
questions for wireless Internet carriers, which have yet to be addressed: 
{\it (i) To what degree can white space bands reduce the network deployment cost of
sparsely populated rural areas as opposed to comparable WiFi-only solutions?} and 
{\it (ii) To what degree can hetergeneous access points benefit the dense 
population areas and sparsely populated rural areas?}

In this paper, we perform a relaxed linear program which considers the 
variation of hetergeneous access point service area too find the lower bound
total number of access points required to serve a given user demand. Further,
we represent an FIXME greedy algorithm to approach the lower bound.  
Across varying hetergeneous white space and WiFi radios combination, population 
densities in representative rural and metropolitan areas we compare the 
cost savings (defined in terms of number of access points reduced) when 
white space bands are not used. We then evaluate our FIXME, showing
the hetergenous band selection across downtown, residential and university settings in
urban area and rural areas and analyze the impact of white space and WiFi
combinations on a wireless deployment in these representative scenarios.

% Paper contributions
% Power efficient
% rural area with non-uniform distribution population time scheduling
% Propogation range





