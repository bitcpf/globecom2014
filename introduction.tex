\section{Introduction}
\label{sec:introduction}

% Deployment constraints: QOS, capacity; coverage; resource constraint
% Problem: Centralize using or distribution
% White space benefit in rural and challenge in populated area
% Issues
% Paper organization


% Background multiband, propagation diversity
The FCC has approved the use of broadband services in the white spaces of 
UHF TV bands, which were formerly exclusively licensed to television broadcasters.
These white space bands are now available for unlicensed public use, enabling the
deployment of wireless access networks across a broad range of scenarios from 
sparse rural areas (one of the key applications identified by the FCC) to dense urban 
areas~\cite{carlson}. The white space bands operate in available channels from 
54-806 MHz, having a far greater propagation range than WiFi bands for similar
transmission power~\cite{balanis2012antenna}. 

Specific to rural areas, the lack of user density and corresponding traffic
demand per unit area as compared to dense urban areas allows greater levels of
spatial aggregation to reduce the total number of required access points, lowering
network deployment costs. In densely populated urban areas, the greater concentration
of users and higher levels of traffic demand can be served by maximizing the spatial
reuse. 
While many works have worked to address multihop wireless network deployment
in terms of maximizing served user demand and/or minimizing network costs,
the unique propagation characteristics and the interference from coexisting
activities in white space bands have either not been jointly studied or assumed to 
have certain characteristics without explicit measurement~\cite{si2010overview}. 
Specifically, previous work has investigated wireless 
network deployment in terms of gateway placement, channel assignment, and 
routing~\cite{he2008optimizing,marina2010topology}.
However, each of these works focus on the deployment in WiFi bands without
considering the white space bands. Moreover, the assumption of idle channels
held in these models fails to match the in-field spectrum utility,
which could degrade the performance of a wireless network. These
two issues are substantial to designing an optimal network deployment and
provide potential commercial wireless services to clients in any location.

Thus, the new opportunities created by white spaces motivate the following 
questions for wireless Internet carriers, which have yet to be addressed: 
{\it (i) To what degree can white space bands reduce the network deployment cost of
sparsely populated rural areas as opposed to comparable WiFi-only solutions?} and 
{\it (ii) Where along the continuum of user population densities do the white
space bands no longer offer cost savings for wireless network deployments?}
In this paper, we perform a measurement study which considers the propagation 
characteristics and observed in-field spectrum availability of white space
and WiFi channels to find the total number of access points required to serve a 
given user demand. Across varying population densities in representative 
rural and metropolitan areas we compare the cost savings (defined in terms of
number of access points reduced) when white space bands are not used.
To do so, we first define the metric to quantify the spectrum utility in a
given measurement location. With the in-field measured spectrum utility data 
in metropolitan and surrounding areas of Dallas-Fort Worth (DFW), we 
calculate the activity level in WiFi and white space bands. Second, we 
propose a measurement-driven framework to find the number of access points required 
for areas with differing population densities according to our measurement locations
and census data. We then evaluate our measurement-driven framework, showing
the band selection across downtown, residential and university settings in
urban area and rural areas and analyze the impact of white space and WiFi
channel combinations on a wireless deployment in these representative scenarios.

% Paper contributions
The main contributions of our work are as follows:
\begin{itemize}
\item We perform in-field measurements of spectrum utilization in various representative
scenarios across the DFW metroplex, ranging from sparse rural to dense urban areas and 
considering the environmental setting (e.g., downtown, residential, or university campus).
%Then we split the area into sub-areas according to the population density 
%and analyze the measurement within sub-areas.
\item W
\item We analyze our framework under capacity and coverage constraints 
\item We quantify the impact of white space and WiFi channel
\end{itemize}


