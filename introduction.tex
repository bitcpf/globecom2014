\section{Introduction}
\label{sec:introduction}

% Background multiband 
% Channel utility
% Traditional hypothesis in previous works
% White space benefit in rural and challenge in populated area
% Issues
% Paper organization

%About a decade ago, numerous cities solicited proposals from
%network carriers for exclusive rights to deploy city-wide WiFi,
%spanning hundreds of square miles.
%While the vast majority of the resulting awarded contracts used
%a wireless mesh topology, initial field tests revealed that the 
%actual WiFi propagation could not achieve the proposed mesh node
%spacing. As a result, many network carriers opted to pay millions of 
%dollars in penalties rather than face the exponentially-increasing
%deployment costs (e.g., Houston~\cite{cnet_aug07} and 
%Philadelphia~\cite{arstechnica_may08}). 
%Thus, while a few mesh networks have been deployed in 
%certain populated communities~\cite{CRSK06,google_imc08},
%wireless mesh networks have largely been unsuccessful in achieving 
%the scale of what was once anticipated~\cite{taps}.
%Both academies and industries are putting efforts on providing 
%Internet connection for metropolitan areas~\cite{pahlavan2011principles}.
 
Numerous cities are pursing city-wide WiFi deployment.
Both academies and industries are putting efforts on providing 
Internet connection for metropolitan areas for this opportunity~\cite{pahlavan2011principles}.
While a few mesh networks have been deployed in 
certain populated communities~\cite{CRSK06,google_imc08},
wireless mesh networks have largely been unsuccessful in achieving 
the scale of what was once anticipated~\cite{taps}. 
As a result, many network carriers opted to pay millions of soolars 
in penalties rather than facing the exponential-increasing deployment
 costs~\cite{cnet_aug07}.
Part of the reasons are the cost of building tons of access points 
and the failure of cooperation with existing wireless instrument.

%http://news.cnet.com/8301-10784_3-9768759-7.html
%http://arstechnica.com/gadgets/2008/05/philadelphias-municipal-wifi-network-to-go-dark/
Around the same time, the digital TV transition created more
spectrum for use with data networks~\cite{fccwhitespace}. These white 
space bands operate in available channels from 54-806 MHz, having
increased propagation characteristics as compared to 
WiFi~\cite{balanis2012antenna}. The large scale communication range 
help the network carriers reducing the access points for covering a 
certain area. Hence, the FCC has identified rural
areas as a key application for white space networks since the reduced
population from major metropolitan areas allows a greater service area
per backhaul device without saturating wireless capacity. 
Naturally, the question arises for these rural communities as well as more dense 
urban settings: {\it how can the emerging white space bands improve 
large-scale mesh network deployments?} 
While much work has been done on deploying wireless networks
the differences in propagation and the activities in white space bands
have not been exploited simultaneously~\cite{si2010overview}.
The existing signals in both WiFi channels and white space band channels 
is an issue for all the topics in wireless network deployment.
The clean channel hypothesis held in previous models fails to match the
in-field spectrum utility, which could slash the performance of mesh 
network.

% Paper topic
In this paper, we leverage the diversity in propagation and spectrum utility 
of white space and WiFi bands in the deployment of large-scale
 wireless mesh networks. 
 To do so, we first from a metric jointly exploit propagation and 
 spectrum utility. Second, we proposed a measurement driven framework 
 for finding the lower bound number of access points for multiple type of
 areas. Then, with in-field measured data in Dallas-Fort Worth area
  we shows the band selection variation in downtown area, neighborhood, 
  university campus, urban area and rural area.
Finally, we quantify the degree to which the joint use of both band types can improve the 
performance of wireless mesh networks in these diverse scenarios.

% Paper contributions
The main contributions of our work are as follows:
\begin{itemize}
\item We develop framework based on in-field measurement to jointly leverage white space and
 WiFi bands to serve the demand in terms of multiple type of areaswireless mesh networks.  
\item We design and perform in-field measurement in multiple type populated areas in 
Dallas-Fortworth metro. Then we apply the measurement in band selection for wireless network
deployment and leverage the diversity in multiple populated areas.
\item We perform extensive analysis across diverse propagation, band availability, Inter-network interference,
demand population fitting for multiple metro areas. 
We additionally show that the joint use of the two types of bands (i.e., WhiteMesh 
networks) can yield up to FIXME saving for wireless mesh network deployment.
\end{itemize}


