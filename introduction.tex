\section{Introduction}
\label{sec:introduction}

% Background multiband 
% Channel utility
% Traditional hypothesis in previous works
% White space benefit in rural and challenge in populated area
% Issues
% Paper organization

% http://www.carlsonwireless.com/rural-connect-press-release.html

The FCC has approved the broadband services in white spaces of UHF TV band,
the frequencies between licensed television broadcasters 
for unlicensed public use, opening up the prospects for
commercial use of the white spaces in rural areas in the US~\cite{carlson}.
The application of white space bands operates in available channels from 
54-806 MHz, having wider propagation range than
WiFi bands~\cite{balanis2012antenna}. 

The wider communication range 
of white space bands can help reduce the number of access points 
in rural areas with low population density. In densely populated urban areas,
the high capacity can be achieved by combining the wide broadcast range 
with high degree of spacial reuse.
Then, the question arises for Internet service carrier
: {\it What is the difference between the impact of the emerging white 
space bands on mesh network deployment in rural areas and urban areas?} 

While many works have been done on deploying wireless networks,
the unic propagation characteristics and the interference from coexisting 
activities in white space bands have not been exploited simultaneously~\cite{si2010overview}.
Previous work has investigated wireless network deployment in
gateway placement, channel assignment, and routing problems~\cite{he2008optimizing,marina2010topology}. 
However, these works focus on the deployment in WiFi bands without 
considering the white space bands. Moreover, the clean channel 
hypothesis held in these models fails to match the in-field spectrum utility, 
which could degrade the performance of the mesh network. These 
two issues are substantial to design a proper network deployment and 
provide potential commercial wireless services to clients in rural area.

% Paper topic
In this paper, we perform a measurement study which consider the propagation 
characteristics and the observed spectrum utility to consider the amount of 
cost savings that a rural network could achieve with the use of white spaces.  
To do so, we first formulate the metric to quantify spectrum utility. 
Second, we proposed a measurement-driven framework to balance the number of 
access points for areas with different population densities. Then, with 
in-field measured spectrum utility data in Dallas-Fort Worth (DFW) area, we 
calculate the activity level in WiFi and white space bands. 
We evaluate our framework with the measurements and shows the band selection 
variation in downtown area, neighborhood, university campus, urban area and 
rural area. Finally, we analyze the impact of channel combination in white 
spaces on the access point deployment for a typical urban area.

% Paper contributions
The main contributions of our work are as follows:
\begin{itemize}
\item In-field spectrum utility measurement is performed in DFW metropolitan.
 Then we analyze the measurement results in multiple dense area types.
\item We develop a measurement driven Multi-band Access Point Estimation framework 
to jointly leverage white space and WiFi bands for serving wireless mesh networks in an arbitrary area.  
\item We perform extensive analysis across diverse propagation, existing spectrum utility under capacity
 and coverage constraints with our measurements in the framework, showing that with white space band, 
 the number of access points outperforms WiFi only deployment by up to 1650\% in sparse rural area.
\item Given certain channel resources, we additionally analyze the combination of channels in multiple bands 
to find the general rules in choosing bands. 
\end{itemize}


