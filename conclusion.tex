\section{Conclusion}
\label{sec:conclusion}
In this paper, we jointly considered the use of WiFi and white space bands for 
%deploying wireless access networks across a broad range of population densities.
%To consider network deployment costs, we proposed a Multiband Access Point Estimation 
%framework to find the number of access points required in a given region.
%We then performed spectrum utilization measurements in the DFW metropolitan 
%and surrounding areas to drive our framework and find the influence of white spaces on
%network costs in these representative areas. Through 
%extensive analysis across varying population density and channel combinations across bands, 
%we show that white space bands can reduce the number of access points by 1650\%
%and 660\% in rural and sparse urban areas, respectively. However, the same cost savings
%are not achieved in dense urban and downtown type area. Finally, we investigate different 
%band combinations in two population densities to show that greater access to white space 
%channels have greater total savings of mesh nodes when the total number of channels used 
%in the network is fixed (i.e., given a total number of allowable WiFi and white space channels). 
%As the population and spectrum utilization increase, the cost savings of white space bands
%diminish to the point that WiFi-only channel combinations can be optimal.


