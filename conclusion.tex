\section{Conclusion}
\label{sec:conclusion}
In this paper, we exploited the joint use of WiFi and white space bands for 
improving the service in multiple type of areas.  To do so, we
propose an multiband coverage frame to find the number of access points in an arbitrary area.
 We then take spectrum activities measurement in DFW metropolitan area and involve the data to 
 our framework to find the best best band combination for different type of areas. Through 
extensive analysis across varying population density, channel availability across bands, 
we show that employing white space band in rural area gain 1650\% reducing the number of 
access points, 660\% in sparse urban area; but fails to get benefit in density urban and downtown type area.
Moreover, we investigate different band combination in sparse urban area, even with high existing
spectrum activities, white space bands still bring benefit for these areas.
%In future work, we will adapt our algorithms to be used with dynamically-changing
%network conditions, in the field on large-scale WhiteMesh networks.

%investigated the channel assignment in multi-band scenario to leverage the propagation incluence for mesh network applications. 
%We have presented the multi-band mesh network architecture, a new defination of path interference over network, and analyze the advantages and disadvantages of white space bands.
%According to the analysis, we formally propose Best Path Selection and Growing Spanning Tree algorithms for channel assignment in multi-band network. Simulation results show that our scheme outperforms the existing scheme substantially.
%Dynamic and distributed algorithms for multi-band channel assignment problems will be of our future work.

