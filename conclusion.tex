\section{Conclusion}
\label{sec:conclusion}
In this paper, we exploited the joint use of WiFi and white space bands for 
improving the service in multiple type of areas.  To do so, we
propose an multiband access point estimation framework to find the number of access points in an arbitrary area.
We then take spectrum utility measurements in DFW metropolitan area and involve the data to 
our framework to find the white spaces influence in different type of areas. Through 
extensive analysis across varying population density, channel combinations across bands, 
we show that employing channel combination with white space band in rural area gains up to 1650\% in the number of 
access points, 660\% in sparse urban area; but fails to get benefit in dense urban and downtown type area.
Moreover, we investigate different band combinations in two population density setup. The numerical results show that 
more white space channels when fix the total number of channels could earn more gains in sparse
area. But as the population and spectrum utility increase, white spaces stop outperforming
mixed white space plus WiFi channels combination.


