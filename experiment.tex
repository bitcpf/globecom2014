\section{Numerical Evaluation and Analysis}
\label{sec:experimentdesign}
To evaluate the performance of hetegenous wireless network deployment, we perform  
numberical evaluation with linear program, and MHAPD algorithm to analyze the role
of white space and WiFi bands in total access points required for a given deployment 
area.

\subsection{Experimental Setup}
% 4 bands, capacity, individual traffic demand,  
In the evaluation, we set the demand request as 2 Mbps per person with the population
density from 20 to 2000 per square kilometer. We assume $30\%$ residents will use this
service, the maximum transmit power is 30 dBm, and a path loss exponent of $3.5$~\cite{meikle2012global}. 

% Channel availability and influence, and in hexagon model
We adopt an 802.11n maximum data rate of 600 Mbps. In the protocol model, the interference 
range is as twice as the communication range. We investigate both traffic demand and the 
number of white space channel influence on hetergeneous wireless network deployment. 
We have interference free scenario, each band has at least 3 channels, which fits for
most rural areas and some cities, such as Houston~\cite{googledatabase}. In this scenario,
it is possible to use all hetergeneous access points since hetergeneous access point could 
serve more area. However, in the field, there are some cities has area only one or two 
licened white space channel, such as Salt Lake City~\cite{googledatabase}. In these scenario, 
only part of the access points could be hetergeneous. We run numerical simulation of
both the scenarios and analyze the hetergeneous access points amount of the results.

We given the target area as $15\times 15$ square kilometers. In the numerical simulation, 
we assign orthogonal WiFi channels in 2.4 GHz,5.8 GHz and white space 
channels in 450 MHz, 800MHz. Then we calculate the service area of access point according to 
their radio combinations as described in~\ref{subsec:problem} with a hexagon model. Then we 
run our linear program and MHAPD mehtods to investigate the benefit from white space band 
and in what degree hetergeneous access point is beter than single radio access point. 


\subsection{Results and Analysis} 
\label{subsec:result}

% Results
Figure~\ref{fig:enoughchannels} shows access point number to serve the target area when 
the area has more than 3 white space channels, which means white space radios could be used
on all access points in hexagon deployment model. In this scenario, at the beginning, the served
area of WiFi only access point is restricted by the communication range. As the population distribution
increase, the served area of WiFi only access point will be limited by the traffic demand instead 
of the communication range. The curve keeps flat untill the traffic demand becomes the limitation
of the served area. In the hetergeneous deployment, the served area is restricted by the traffic
demand at the beginning, the number of access point increase as the traffic demand increase. 
Also since there are enough channels can be reused, our algorithm use almost the same number of 
access point to serve the target area. 






\begin{figure}
%\vspace{-0.0in}
\centering
\includegraphics[width=74mm]{figures/enoughchannels}
\vspace{-0.1in}
\caption{Sufficient White Space Channels Scenario}                                                                 
\label{fig:enoughchannels}
\vspace{-0.1in}
\end{figure}



\begin{figure}
%\vspace{-0.0in}
\centering
\includegraphics[width=74mm]{figures/onewhitechannel}
\vspace{-0.1in}
\caption{One white channel}                                                                 
\label{fig:onewhitechannel}
\vspace{-0.1in}
\end{figure}


\begin{figure}
%\vspace{-0.0in}
\centering
\includegraphics[width=74mm]{figures/percentage}
\vspace{-0.1in}
\caption{Percentage of Heterogeneous Access Points}                                                                 
\label{fig:heappercentage}
\vspace{-0.1in}
\end{figure}

% Analysis
