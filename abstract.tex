\begin{abstract}

%Then, we start in the abstract with the story that a network deployment and corresponding cost structure 
%in a dense urban area should look very different from that in a rural area.  Mention that white 
%spaces allow the degree to which such scalability can truly be achieved due to the higher degree of 
%propagation, allowing greater levels of aggregation.  This is in direct contrast to urban areas which 
%try to maximize spatial reuse.   Then, moving to the introduction, the focus is not on metropolitan 
%deployment but rather rural deployment.  You can cite some announcements by the FCC or federal 
%government of the intention of the white space bands (rural areas).  You can then say that we perform 
%a measurement study which considers these propagation characteristics and observed spectrum utility 
%to consider the amount of cost savings a rural network could achieve with the use of white spaces. 
%Then, all of the huge gains you show now really support this story. 
%

While many metropolitan areas sought to deploy city-wide WiFi networks, the densest urban areas were not
able to broadly leverage the technology for large-scale Internet access.  Ultimately, the small 
spatial separation required for effective 802.11 links in these areas resulted in prohibitively large up-front 
costs.  The FCC has reapportioned spectrum from TV white spaces for the purposes of large-scale Internet 
connectivity via wireless topologies of all kinds.
The far greater range of these lower carrier frequencies are especially critical in rural areas, where
high levels of aggregation could dramatically lower the cost of deployment and is in direct contrast 
to dense urban areas, in which networks are built to maximize spatial reuse.  Thus, leveraging a broad 
range of spectrum across diverse population densities becomes a critical issue for the deployment of data
networks with WiFi and white space bands. In this paper, we measure the spectrum utility in the Dallas-Fort Worth metropolitan and surrounding
areas and propose a measurement-driven band selection framework, Multiband Access Point Estimation (MAPE). 
In particular, we study the white space and WiFi bands with in-field spectrum utility measurements, revealing 
the number of access points required for an area with channels in multiple bands. In doing so, we find that 
networks with white space bands reduce the number of access points up to 1650\% in sparse
rural areas over similar WiFi-only solutions. In more populated rural areas and sparse urban areas, we 
find an access point reduction of 660\% and 412\%, respectively.  However, due to the heavy use of white
space bands in dense urban areas, the cost reductions invert (an increase in required access points 
of 6\%).  Finally, we numerically analyze band combinations in typical rural and urban areas and show 
the critical factor that leads to cost reduction: considering the same total number of channels, as more 
channels are available in the white space bands, less access points are required for a given area.

%Wireless Mesh Network is an efficient and low cost solution for large-scale Internet connectivity.
%A network deployment and corresponding cost structure in a dense urban area should look very different
%from that in a rural area. FCC intends to improve the coverage of rural area in U.S. by reapportioning 
%spectrum for data communication with far greater range in lower carrier frequencies. These white 
%spaces allow the degree to which such scalability can truly be achieved due to the higher degree of 
%propagation, allowing greater levels of aggregation. The large propagation range is direct contrast to 
%urban areas which try to maximize spatial reuse. Thus, leveraging the spectrum across diverse
%population densities and finding the proper bands become an urgent task for the deployment of large-scale 
%mesh access networks. In this work, we measure the spectrum utility in DFW area and propose a measurement 
%driven band selection framework Multiband Access Point Estimation (MAPE) which explores the white space
%bands with in-field spectrum utility measurements telling the number of access points required of an area 
%with channels in multiple bands. In doing so, we find the gains of white space bands application over 
%existing WiFi only application in reducing the number of access points up to 1650\% in sparse 
%rural area, 660\% in rural area, and 412\% in sparse urban area. Moreover, due to the heavy used white 
%space bands, the spacial reusable WiFi bands gains 6.25\% reducing the number of access points in 
%urban area, and 6.6\% in downtown area. We use our MAPE framework numerical analyze band combination 
%in typical urban area and the result shows the more channels in white space band, the less access points 
%are required for an arbitrary areas with the same number of channels in total.  


%To deploy new wireless mesh network, knowing the existing signals in the air is a 
% must. The existing signal information is important for all the problems in wireless mesh network, such as
% gateway deployment, channel assignment, and routing. Many work has been done to approach the interference model.
% However, there is no model fit for all metropolitan cities. Moreover, FCC regulations have reapportioned spectrum for 
% data communication with far greater range than WiFi in lower carrier frequencies. 
% Thus, leveraging the spectrum utility and finding the proper bands become an
%  urgent task for metropolitan cities who want to deploy wireless mesh network.
% In this work, we measure the spectrum activities in DFW area and propose a measurement 
% driven band selection framework Multiband Access Point Estimation (MAPE) which explore the white space
% bands with in-field spectrum utility telling the number of access points required to serve an area
% with channels in different bands.
% In doing so, we find gains of white space bands application over existing multi-channel, 
% multi-radio application in reducing the number of access points up to 1650\% in sparse 
% rural area, 660\% in rural area, and 412\% in sparse urban area.
% Moreover, due to the heavy used white space band, the spacial reusable WiFi bands gains 6.25\% reducing 
% the number of access points in urban area, and 6.6\% in downtown area.
% Moreover, we use our MAPE framework numerical analyze band combination in typical urban area and the result
% shows with the same number of channels, the more channels in white space band, the less access points needs for
% a candidate areas.  
\end{abstract}
