\begin{abstract}
Wireless Mesh Network is an efficient and low cost solution for large-scale Internet connectivity
in metropolitan areas. To deploy new wireless mesh network, knowing the existing signals in the air is a 
must. The existing signal information is important for all the problems in wireless mesh network, such as
gateway deployment, channel assignment, and routing. Many work has been done to approach the interference model.
However, there is no model fit for all metropolitan cities. Moreover, FCC regulations have reapportioned spectrum for 
data communication with far greater range than WiFi in lower carrier frequencies. 
Thus, leveraging the spectrum utility and finding the proper bands become an
 urgent task for metropolitan cities who want to deploy wireless mesh network.
In this work, we measure the spectrum activities in DFW area and propose a measurement 
driven band selection framework which explore the white space
bands with in-field spectrum utility telling the number of access points required to serve an area
with channels in different bands.
In doing so, we find gains of white space bands application over existing multi-channel, 
multi-radio application up to 1650\% in sparse rural area, 660\% in rural area, and 412\% in sparse urban area.
Moreover, due to the heavy used white space band, the spacial reusable WiFi bands gains 6.25\% in urban area, and
6.6\% in downtown area.
Moreover, we use our framework numerical analyze band combination in typical urban area and the result
shows with the same number of channels, the more channels in white space band, the less access points needs for
a candidate areas.  
\end{abstract}
